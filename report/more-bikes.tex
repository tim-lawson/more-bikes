\documentclass[11pt]{extarticle}

% Page layout
\usepackage{geometry}
\geometry{
  a4paper,
  margin={1in,1in},
}
\usepackage{setspace}

% Typography
% https://ctan.math.washington.edu/tex-archive/fonts/etbb/doc/ETbb-doc.pdf
\usepackage[p]{ETbb}
\usepackage[scaled=.95,type1]{cabin}
\usepackage[varqu,varl]{zi4}
\usepackage[T1]{fontenc}
\usepackage[libertine,vvarbb]{newtxmath}
\usepackage{microtype}
\frenchspacing

% Bibliography
\usepackage[
  style=authoryear-ibid,
  maxcitenames=2,
  doi=false,
  isbn=false,
  % Omit URL except for @online
  url=false,
]{biblatex}

\addbibresource{~/library.bib}

% Map @misc to @online
\DeclareSourcemap{
  \maps[datatype=bibtex, overwrite=true]{
    \map{
      \step[typesource=misc, typetarget=online]
    }
  }
}

\AtEveryBibitem{%
  \clearlist{language}%
  \clearfield{month}%
  \clearfield{note}%
  \clearfield{eprint}%
  % Omit "Visited on <date>"
  \iffieldundef{urlyear}
    {}
    {\clearfield{urlyear}}
}

% Quotations
\usepackage{csquotes}

% Input/include relative paths
\usepackage{import}

% References
\usepackage{hyperref}
\usepackage{cleveref}

% Math
\usepackage{amsmath}
\usepackage{mathtools}

% Tables and charts
\usepackage{caption}
\usepackage{subcaption}
\usepackage{array}
\usepackage{makecell}
\usepackage{multirow}
\usepackage[draft]{graphicx}

\def\imagebox#1#2{\vtop to #1{\null\hbox{#2}\vfill}}
\newcolumntype{R}[1]{>{\raggedleft\arraybackslash }b{#1}}
\newcolumntype{L}[1]{>{\raggedright\arraybackslash }b{#1}}
\newcolumntype{C}[1]{>{\centering\arraybackslash }b{#1}}

\usepackage{tikz}
\usetikzlibrary{external}
\tikzexternalize[prefix=figures/]

\usepackage{pgfplots}
\usepackage{pgfplotstable}
\pgfplotsset{compat=1.17}
\usepgfplotslibrary{colorbrewer}
\usepgfplotslibrary{fillbetween}
\usepgfplotslibrary{groupplots}

% Exclusive filter
\pgfplotsset{
  discard if/.style 2 args={
      x filter/.append code={
          \readlist\mylist{#2}%
          \foreachitem\z\in\mylist[]{%
            \ifdim\thisrow{#1} pt=\z pt
              \def\pgfmathresult{inf}
            \fi
          }
        }
    },
}

% Inclusive filter
\pgfplotsset{
  discard if not/.style 2 args={
      x filter/.code={
          \edef\tempa{\thisrow{#1}}
          \edef\tempb{#2}
          \ifx\tempa\tempb
          \else
            \def\pgfmathresult{inf}
          \fi
        }
    }
}

% Comments
\setlength{\marginparwidth}{1in}
\usepackage{todonotes}
\newcounter{todocounter}
\newcommand{\todonum}[1]{%
  \stepcounter{todocounter}%
  \todo[color={red!100!green!33},inline,size=\small]{
    \thetodocounter: #1
  }%
}

\begin{document}

\title{More Bikes: Experiments in Univariate Regression}
\author{Tim Lawson}
\date{\today}

\maketitle

\section{Task description}

The assignment is to predict the number of available bikes at 75 rental stations in
three hours' time for a period of three months, beginning in November 2014, i.e., a
supervised univariate regression problem.
It is divided into three sub-tasks, which differ in the information that is available:
\begin{itemize}
  \item \textbf{Sub-task 1}.
        The number of available bikes at each of the 75 stations for the month of October 2014.
        This sub-task may be approached by building a separate model for each station or a
        single model for all 75 stations.
  \item \textbf{Sub-task 2}.
        A set of linear models that were trained on the number of available bikes at each of a
        separate set of 200 stations for a year.
        For the first ten stations, this data is available for analysis but not training.
  \item \textbf{Sub-task 3}.
        Both of the above.
\end{itemize}
Sub-tasks 2 and, optionally, 3 require the use of ensemble methods.
The predictions are evaluated by the mean absolute error (MAE) between the predicted
and true numbers of available bikes over the period of three months, beginning in
November 2014.
The evaluation data is not available to participants but the score achieved on a
held-out test set is reported on the task leaderboard.
This report begins with a preliminary analysis of the data and then describes the
approach taken to each sub-task and the cross-validation results obtained.

\section{Data analysis}

The data is recorded at hourly intervals; a summary of its features is given in
\cref{table:features}.
The `station' features are constant for all instances at a given station and the
meteorological features are constant for all instances at a given timestamp.
Hence, the variances of the `station' features are zero for the first case of sub-task
1.
The `profile' features, i.e., the features derived from the numbers of available bikes
at preceding times are not defined for the first week of instances at each station.
Naturally, the number of bikes available at a given station is bounded by zero and the
number of docks at that station.
Additionally, the variance of the \texttt{precipitation} feature is zero for both cases
of sub-task 1.

The following analysis is based on the combination of the available data for sub-task 1
and the first ten stations in sub-task 2.
It is assumed that the distributions of the feature values are representative of those
in the evaluation data.

\import{.}{figure-table-features.tex}
\import{.}{figure-chart-weekday.tex}
% \import{.}{figure-chart-weekday-separate.tex}
% \import{.}{figure-chart-distributions.tex}

\subsection{Feature selection}

\paragraph{Variance}

Zero-variance features were removed automatically from the data.

\begin{itemize}
  \item Which features are most informative?
  \item Describe the distributions of the features and their correlations.
  \item Describe the distributions of the fraction of available bikes in terms of
        temporal features and the possible derived features.
\end{itemize}

\section{Methodology}

Throughout this report, ten-fold `forward chaining' or nested cross-validation is
performed, which is illustrated in \cref{fig:chart-cross-validation}.
Generally, standard $k$-fold cross-validation is disfavoured for time-series data due
to the inherent correlation between successive folds \parencite{Bergmeir2018}.

\begin{itemize}
  \item Hyperparameter search.
  \item Evaluation metric.
  \item Statistical significance tests (paired $t$-tests).
\end{itemize}

\import{.}{figure-chart-cross-validation.tex}

\printbibliography

\end{document}
