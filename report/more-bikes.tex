\documentclass[11pt]{extarticle}

% Page layout
\usepackage{geometry}
\geometry{
  a4paper,
  margin={1in,1in},
}
\usepackage{setspace}

% Typography
% https://ctan.math.washington.edu/tex-archive/fonts/etbb/doc/ETbb-doc.pdf
\usepackage[p]{ETbb}
\usepackage[scaled=.95,type1]{cabin}
\usepackage[varqu,varl]{zi4}
\usepackage[T1]{fontenc}
\usepackage[libertine,vvarbb]{newtxmath}
\usepackage{microtype}
\frenchspacing

% Bibliography
\usepackage[
  style=authoryear-ibid,
  maxcitenames=2,
  doi=false,
  isbn=false,
  % Omit URL except for @online
  url=false,
]{biblatex}

\addbibresource{~/library.bib}

% Map @misc to @online
\DeclareSourcemap{
  \maps[datatype=bibtex, overwrite=true]{
    \map{
      \step[typesource=misc, typetarget=online]
    }
  }
}

\AtEveryBibitem{%
  \clearlist{language}%
  \clearfield{month}%
  \clearfield{note}%
  \clearfield{eprint}%
  % Omit "Visited on <date>"
  \iffieldundef{urlyear}
    {}
    {\clearfield{urlyear}}
}

% Quotations
\usepackage{csquotes}

% Input/include relative paths
\usepackage{import}

% References
\usepackage{hyperref}
\usepackage{cleveref}

% Math
\usepackage{amsmath}
\usepackage{mathtools}

% Tables and charts
\usepackage{caption}
\usepackage{subcaption}
\usepackage{array}
\usepackage{makecell}
\usepackage{multirow}
\usepackage[draft]{graphicx}

\def\imagebox#1#2{\vtop to #1{\null\hbox{#2}\vfill}}
\newcolumntype{R}[1]{>{\raggedleft\arraybackslash }b{#1}}
\newcolumntype{L}[1]{>{\raggedright\arraybackslash }b{#1}}
\newcolumntype{C}[1]{>{\centering\arraybackslash }b{#1}}

\usepackage{tikz}
\usetikzlibrary{external}
\tikzexternalize[prefix=figures/]

\usepackage{pgfplots}
\usepackage{pgfplotstable}
\pgfplotsset{compat=1.17}
\usepgfplotslibrary{colorbrewer}
\usepgfplotslibrary{fillbetween}
\usepgfplotslibrary{groupplots}

% Exclusive filter
\pgfplotsset{
  discard if/.style 2 args={
      x filter/.append code={
          \readlist\mylist{#2}%
          \foreachitem\z\in\mylist[]{%
            \ifdim\thisrow{#1} pt=\z pt
              \def\pgfmathresult{inf}
            \fi
          }
        }
    },
}

% Inclusive filter
\pgfplotsset{
  discard if not/.style 2 args={
      x filter/.code={
          \edef\tempa{\thisrow{#1}}
          \edef\tempb{#2}
          \ifx\tempa\tempb
          \else
            \def\pgfmathresult{inf}
          \fi
        }
    }
}

% Comments
\setlength{\marginparwidth}{1in}
\usepackage{todonotes}
\newcounter{todocounter}
\newcommand{\todonum}[1]{%
  \stepcounter{todocounter}%
  \todo[color={red!100!green!33},inline,size=\small]{
    \thetodocounter: #1
  }%
}

\usepackage{siunitx}

\usepackage{listings}
\lstset{
    basicstyle=\tt,
}
\newcommand{\sklearn}[1]{
  \href{https://scikit-learn.org/stable/modules/generated/sklearn.#1.html}{\lstinline|sklearn.#1|}
}

\newcommand{\isholiday}{is\_holiday}
\newcommand{\windspeedmax}{wind\_speed\_max}
\newcommand{\windspeedavg}{wind\_speed\_avg}
\newcommand{\winddirection}{wind\_direction}
\newcommand{\bikesavgfull}{bikes\_avg\_full}
\newcommand{\bikesavgshort}{bikes\_avg\_short}
\newcommand{\bikesh}{bikes\_3h}
\newcommand{\bikeshdiffavgfull}{bikes\_3h\_diff\_avg\_full}
\newcommand{\bikeshdiffavgshort}{bikes\_3h\_diff\_avg\_short}

\begin{document}

\title{More Bikes: Experiments in Univariate Regression}
\author{Tim Lawson}
\date{\today}

\maketitle

\section{Task description}

The assignment is to predict the number of available bikes at 75 rental stations in
three hours' time for a period of three months, beginning in November 2014, i.e., a
supervised univariate regression problem.
It is divided into three sub-tasks, which differ in the information that is available:
\begin{itemize}
  \item \textbf{Sub-task 1}.
        The number of available bikes at each of the 75 stations for the month of October 2014.
        This sub-task may be approached by building a separate model for each station or a
        single model for all 75 stations.
  \item \textbf{Sub-task 2}.
        A set of linear models that were trained on the number of available bikes at each of a
        separate set of 200 stations for a year.
        For the first ten stations, this data is available for analysis but not training.
  \item \textbf{Sub-task 3}.
        Both of the above.
\end{itemize}
Sub-tasks 2 and, optionally, 3 require the use of ensemble methods.
The predictions are evaluated by the mean absolute error (MAE) between the predicted
and true numbers of available bikes over the period of three months, beginning in
November 2014.
The evaluation data is not available to participants but the score achieved on a
held-out test set is reported on the task leaderboard.
This report begins with a preliminary analysis of the data and then describes the
approach taken to each sub-task and the cross-validation results obtained.

\section{Data analysis}
\label{sec:data-analysis}

The data is recorded at hourly intervals.
A summary of the features is given in \cref{table:features}.
Notably, the meteorological features are constant for all stations at a given timestamp.
The `profile' features, i.e., the features derived from the numbers of available bikes
at preceding times, are not defined for the first week of instances at each station,
except for \texttt{bikes\_3h}.
Naturally, the number of available bikes at a given station is bounded by zero and the
number of docks at that station.
The variances and pairwise correlations of the features are described in \cref{sec:feature-selection}.

\import{.}{figure-table-features.tex}
% \import{.}{figure-chart-weekday.tex}
% \import{.}{figure-chart-weekday-separate.tex}
% \import{.}{figure-chart-distributions.tex}

\section{Methods}
\label{sec:methods}

Throughout this report, I used \texttt{scikit-learn} to conduct experiments
\parencite{Pedregosa2011}.
In each case, preprocessing and feature selection were performed by \emph{estimators}
that implemented the \emph{transformer} interface; prediction was performed by
estimators that implemented the \emph{predictor} interface; and estimators were
composed into \texttt{Pipeline} objects over which hyperparameter search was performed
\parencite[4-9]{Buitinck2013}.
The bounds on the number of available bikes at a given station were enforced by
predicting the \emph{fraction} of bikes, i.e., the number of bikes divided by the number
of docks at the station.
This was implemented by an extension of the \texttt{TransformedTargetRegressor}
meta-estimator to permit data-dependent
transforms\footnote{\sklearn{compose.TransformedTargetRegressor}}.

Generally, standard $k$-fold cross-validation is disfavoured for time-series data due
to the inherent correlation between successive folds \parencite{Bergmeir2018}.
Instead, nested time-series
cross-validation\footnote{\sklearn{model\_selection.TimeSeriesSplit}} with ten folds
was performed to evaluate the models.
This behaviour is illustrated in \cref{fig:chart-cross-validation}.

\begin{itemize}
  \item Hyperparameter search.
  \item Evaluation metric.
  \item Statistical significance tests (paired $t$-tests).
\end{itemize}

\import{.}{figure-chart-cross-validation.tex}

\subsection{Feature selection}
\label{sec:feature-selection}

In each case, zero-variance features were automatically
excluded\footnote{\sklearn{feature\_selection.VarianceThreshold}} because they are
individually uninformative.
The available data for sub-task 1 is limited to the month of October 2014; therefore,
the month and year were excluded.
Additionally, the `station' features (\cref{table:features}) are constant for all
instances at a given station; hence, for the first case of sub-task 1, the variances of
these features are zero.
Finally, the \texttt{precipitation} feature is zero for all instances.

To determine which features are redundant, i.e., uninformative in combination, the
Pearson correlation coefficients between pairs of quantitative features were computed
(\cref{fig:chart-correlations}).
This analysis yielded the following observations:
\begin{itemize}
  \item \texttt{\bikeshdiffavgfull} and \texttt{\bikeshdiffavgshort} are fully correlated ($r = 1.00$).
  \item \texttt{\bikesavgfull} and \texttt{\bikesavgshort} are fully correlated ($r = 1.00$).
  \item \texttt{\windspeedmax} and \texttt{\windspeedavg} are highly correlated ($r = 0.96$).
\end{itemize}
Hence, the second of each of these pairs of features was manually excluded.

\import{.}{figure-chart-correlations.tex}

\todonum{Describe the distributions of the fraction of available bikes in terms of
  temporal features and the possible derived features.}

\printbibliography

\end{document}
