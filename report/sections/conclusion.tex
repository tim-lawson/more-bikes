\section{Conclusion}
\label{sec:conclusion}

\begin{table}
  \centering
  \import{sub-task-2}{tab-t-tests-all}
  \caption{
    The $t$-statistics from paired $t$-tests, and $p$-values from the Nemenyi test, on
    the mean scores of the best estimators for sub-tasks~1~(b) and 2.
    As elsewhere, a positive $t$-statistic indicates that `Model~B' achieved a lower mean
    score than `Model~A'.
  }
  \label{sub-task-2:tab-t-tests-all}
\end{table}

Finally, I performed the same comparison for the best estimators for sub-task~1~(b) and
sub-task~2 (\cref{sub-task-2:tab-t-tests-all}).
The results were broadly consistent with those in
\cref{sub-task-1:tab-t-tests-b,sub-task-2:tab-t-tests}, i.e., the following model classes
performed significantly better than the baseline but not significantly better than
each other:
\begin{itemize}
  \item sub-task~1~(b), gradient-boosted decision tree;
  \item sub-task~2, ridge regression; and
  \item sub-task~2, gradient-boosted decision tree.
\end{itemize}
Among these, the first performed best on the held-out test set
(\cref{sub-task-1:tab-results-b,sub-task-2:tab-results}).
I investigated whether combining the best estimators would improve the score on the
held-out test set, the results failed to improve on the best score of $2.15$, which
thus formed the final submission.
