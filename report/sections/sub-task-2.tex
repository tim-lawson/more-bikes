
\section{Sub-task 2}
\label{sec:sub-task-2}

\subsection{Model classes}
\label{sec:sub-task-2:model-classes}

The available information for sub-task~2 is a set of linear models that were trained on
the data from a separate set of 200 stations.
There are six models for each station, each of which uses a different set of features,
listed in \cref{sub-task-2:tab-features-rlm}.
Unlike the gradient-boosted decision tree implementation in \texttt{scikit-learn},
these models do not support missing values, so I imputed the missing values of
quantitative features with the mean value of the feature over the data provided for
sub-task~1.
I used \sklearn{ensemble}{StackingRegressor} to combine the predictions of the models
by stacked generalization \parencite{Wolpert1992}.
This model class is a meta-estimator that trains a second-level regressor on the
predictions of a set of first-level regressors -- in this case, the pre-trained models.

\begin{table}
  \centering
  \import{sub-task-2}{tab-features-rlm}
  \caption{
    The features used by the different kinds of pre-trained linear models for sub-task~2.
    The features are described in \cref{tab-features} and follow the same ordering.
  }
  \label{sub-task-2:tab-features-rlm}
\end{table}

\begin{figure}
  \centering
  \import{sub-task-2}{fig-box-plot}
  \caption{
    A box plot of the mean scores achieved by individual pre-trained models for
    sub-task~2 on the ten cross-validation folds of the data from the 75 stations for
    the month of October 2014, i.e., the same data as for sub-task~1~(b).
    The kinds of models are listed in \cref{sub-task-2:tab-features-rlm}.
  }
  \label{sub-task-2:fig-box-plot}
\end{figure}

For this sub-task, I excluded zero-variance and highly correlated features, but
retained the `profile' features (\cref{sec:data-analysis:feature-selection}).
The \texttt{full} and \texttt{short} features are perfectly correlated, and the latter
were excluded for sub-task~1, so I first compared the different kinds of pre-trained
models.
\Cref{sub-task-2:fig-box-plot} indicates that individual \rlmshort{} and \rlmshorttemp{} models
generally achieved lower mean scores.
However, with the default second-level regressor (ridge regression), stacked
generalization performed very similarly with each kind of model, so I included all the
models in the final analysis.
For the second-level regressor, I evaluated the default, decision trees, and
gradient-boosted decision trees, as in \cref{sec:sub-task-1:model-classes}.
The best estimators were again determined by grid search with ten-fold
cross-validation.

\subsection{Results}
\label{sec:sub-task-2:results}

The mean scores achieved by the best estimators for each model class on the
cross-validation folds are listed in \cref{sub-task-2:tab-results}.
All the model classes achieved a lower score than the baselines in sub-task~1.
As in \cref{sec:sub-task-1}, I performed paired $t$-tests and the Nemenyi test on the
scores on the cross-validation folds to determine whether the differences between the
mean scores achieved by the best estimators were statistically significant
(\cref{sub-task-2:tab-t-tests}).
Stacked generalization with ridge regression or a gradient-boosted decision tree as the
second-level regressor performed significantly better than with a decision tree, but
they did not perform significantly better than each other.

\begin{table}
  \centering
  \import{sub-task-2}{tab-results}
  \caption{
    The mean scores and variances achieved by the best estimators for each model class
    for sub-task~2 on the data provided for sub-task~1, and the corresponding score on
    the held-out test set (\cref{sec:task-description}).
    The best scores on each dataset are underlined.
  }
  \label{sub-task-2:tab-results}
\end{table}

\begin{table}
  \centering
  \import{sub-task-2}{tab-t-tests}
  \caption{
    The $t$-statistics from paired $t$-tests, and $p$-values from the Nemenyi test, on
    the mean scores of the best estimators for sub-task~2.
    As in \cref{sub-task-1:tab-t-tests}, a positive $t$-statistic indicates that `Model B'
    achieved a lower mean score than `Model A'.
  }
  \label{sub-task-2:tab-t-tests}
\end{table}
