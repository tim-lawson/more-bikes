\section{Approach}
\label{sec:approach}

I used the \texttt{scikit-learn} Python package \parencite{Pedregosa2011} throughout
this report.\footnote{The underlying code is available at
  \github{https://github.com/tslwn/more-bikes}{tslwn/more-bikes}.
}
Preprocessing and feature selection were performed by \emph{estimators} that
implemented the \emph{transformer} interface, prediction was performed by estimators
that implemented the \emph{predictor} interface, and estimators were composed into
\texttt{Pipeline} objects over which hyperparameter search was performed
\parencite[4-9]{Buitinck2013}.

Generally, standard $k$-fold cross-validation is disfavoured for time-series data due
to the inherent correlation between successive folds \parencite{Bergmeir2018}.
Instead, I used nested time-series cross-validation\footnote{See
  \sklearn{model\_selection}{TimeSeriesSplit}.
} with ten folds, which is illustrated in
\cref{fig-cv}.
I determined the best estimator for each model class by grid search\footnote{See
  \sklearn{model\_selection}{GridSearchCV} and
  \skl{model\_selection}{HalvingGridSearchCV}.
} with the mean absolute error as the scoring function, following the task description.
Finally, I assessed the statistical significances of the differences between the mean
scores of the best estimators by paired $t$-tests and Nemenyi tests on the scores on
the cross-validation folds \parencite[353-354]{Flach2012}.
I describe these methods in more detail in \cref{sec:sub-task-1,sec:sub-task-2}.

\begin{figure}
  \import{approach}{fig-time-series-split}
  \caption{A visualization of the nested time-series cross-validation behaviour that I
    used, after the documentation for \sklearn{model\_selection}{TimeSeriesSplit}.
  }
  \label{fig-cv}
\end{figure}
